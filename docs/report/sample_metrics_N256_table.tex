
% Auto-generated from results/sample_metrics_N256.csv (N=256)
\begin{table}[H]
    \centering
    \small
    \setlength{\tabcolsep}{4.5pt}
    \renewcommand{\arraystretch}{1.15}
    \begin{tabular}{@{}lrrrrrrrr@{}}
    \toprule
    \textbf{Method} & \textbf{finite} & \textbf{in-range} & \textbf{sat.} & \textbf{TV} & \textbf{L2} & \textbf{var} & \textbf{mean|x|} & \textbf{std} \\
    \midrule
    DAE (Vincent) & 100.0 & 81.2 & 20.2 & 1.515 & 27.73 & 0.489 & 0.667 & 0.741 \\
    $L=1$ & 100.0 & 86.5 & 16.6 & 1.491 & 27.05 & 0.465 & 0.626 & 0.712 \\
    $L=10$ & 92.2 & 60.7 & 64.0 & 0.271 & 22.81 & 0.337 & 0.874 & 0.702 \\
    $L=50$ & 100.0 & 55.6 & 77.1 & 0.160 & 15.46 & 0.161 & 0.982 & 0.430 \\
    NCSN ($L=10$) & 100.0 & 61.6 & 68.8 & 0.251 & 21.41 & 0.296 & 0.955 & 0.636 \\
    \bottomrule
    \end{tabular}
    \caption{\textbf{MNIST sampling diagnostics (quantitative, $N=256$).}
    Percentages are computed over all generated pixels; \textbf{TV} is total variation (lower is smoother);
    \textbf{L2} is mean pairwise $\ell_2$ distance between samples (higher indicates more diversity);
    \textbf{var} is mean pixel variance across samples.}
    \label{tab:mnist_metrics_256}
    \end{table}

    